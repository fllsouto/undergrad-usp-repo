\documentclass[a4paper,11pt]{article}
\usepackage[T1]{fontenc}
\usepackage[utf8]{inputenc}
\usepackage{lmodern}
\usepackage[brazil]{babel}
\usepackage{listings}
\lstset{language=C}
\usepackage{tikz}
\usepackage{graphicx}
\usepackage{amsmath}
\usepackage{amsthm}
\usepackage{amsfonts}
\begin{document}

\begin{center}{\Large \bf Relatorio EP 1 \\ Limiar de Conexidade para certo Grafos Geométricos}\end{center}
\begin{center}{Fellipe Souto Sampaio\footnote{e-mail: fellipe.sampaio@usp.com}}\end{center}

\begin{center}
MAC 0323 Estrutura de Dados \\
Prof. Dr. Yoshiharu Kohayakawa   \\
             
\end{center}


\begin{center}
Instituto de Matemática e Estatística - IME USP \\
 Rua do Matão 1010 \\
 05311-970\, Cidade Universitária, São Paulo - SP \\
\end{center}

\newpage
\section{Client.c}
  Este é o arquivo principal do programa, o qual contêm as funções \textit{int main} e \textit{getArgv}. Explicarei sucintamente seu funcionamento, assim como dos próximos módulos e funções que compoem o programa. 

\subsection{int main}

\textbf{Parâmetros de entrada :} -Nxxx, -sxxx, -dx.xx, -Mxxx, -v, -V, -D, -C. \\
\textbf{Saída:} 0 se a execução ocorrer sem falha.\\

\textbf{Descrição:}
O programa funciona em três modos de execução basicamente: 

\begin{enumerate}
\item[-] Teste de conectividade para um dado N, s e d.
\item[-] Cálculo da densidade normalizada crítica.
\item[-] Busca do menor d, tal que o grafo seja conexo.
\end{enumerate}

Dado uma entrada na chamada do programa através da linha de comando o programa verifica e executa a rotina requisitada pelo usuário.

\subsection{getArgv}

\textbf{Parâmetros de entrada :} argc, argv[ ]. \\
\textbf{Saída:} N, d, s, v, C, m.\\

\textbf{Descrição:}
Le as strings contidas em argv[argc] e procura os padrões de funcionamento do programa, atribuindo os valores fornecidos as suas respectivas variáveis.

\section{Instance.c}
Módulo que concentra as principais funções de instanciação do programa.

\subsection{searchDensity}
\textbf{Parâmetros de entrada :} M, N, out. \\
\textbf{Saída:} Informa na tela ou em arquivo de saída (dependendo do valor de out) o valor do limiar de conexividade da instância M $_i$, com $ 0 \ge i \ge M $, a densidade normalizada critica de M $_i$ e a média das densidades para (N,M).\\

\textbf{Descrição:}
A função recebi suas diretivas de funcionamento e chama a função lessConnectivity para localizar o d$_i*$. C$_i*$ é cálculado e seu valor é salvo em um vetor. Após o fim da execução de todas instâncias seus resultados são somados e divididos por M.

\subsection{lessConnectivity}
\textbf{Parâmetros de entrada :} N. \\
\textbf{Saída:} O menor d tal que o grafo é conexo.\\

\textbf{Descrição:}
A função cria uma nova instância com N pontos e começa a verificação de conexidade com um valor \textit{edge default} de 0.5. O algoritmo adotado é similar a uma busca binária, caso o grafo seja conexo com um dado d este é multiplicado por 0.9, estreitando o valor de busca. Caso seja conexo ainda o menor valor é considerado como teto, e caso o grafo deixe de ser conexo d é multiplicado por 0.05 e somado com si mesmo. O maior valor de não conexidade é salvo e atua como chão, estreitando assim a busca do limiar de conexidade entre um teto e um chão.

O algoritmo é repetido diversas vezes até que a diferentença entre teto e chão seja no máximo de 0.0021, e com isso o valor retornado é o menor d encontrado desde que ele seja conexo.

\subsection{newGraph}
\textbf{Parâmetros de entrada :} N, v. \\
\textbf{Saída:} Retorna TRUE ou FALSE, dependendo da conexidade do grafo.\\

\textbf{Descrição:}
A função cria uma nova instância Grafo, N pontos aleatórios e imprime em arquivo os pontos gerados se for requisitado pelo usuário. Ela utiliza da função searchEdge para testar conexidade da instância criada e retorna o valor lógico do teste em conex.

\section{Creation.c}
  Este módulo agrupa as principais funções de criação dos elementros que compõem uma instância Grafo, como seus pontos e suas flags de iniciali-zação. Os pontos são criados aleatóriamente com o uso da funcão rand() e uma flag de inicialização é setada para cada par ordenado. Esta flag é utilizada para informar se um dado ponto p$_i$ ja foi verificado pelo algoritmo de conexidade. Ainda neste módulo existe a função reset flag, que trabalha em conjunto com a função lessConnectivity, resetando as flags de uma instância a cada nova execução do algoritmo.

\subsection{createList}
\textbf{Parâmetros de entrada :} N. \\
\textbf{Saída:} Retorna um ponteiro para vetor de structs do tipo point de tamanho N.\\

\textbf{Descrição:}
Dado uma configuração fornecida pelo usuário o programa cria um vetor de struct point de tamanho N e atribui N pares ordenados para posições de K$_0$ \ldots K$_{n-1}$ . Se o usuário preferir digitar os pontos é perguntado qual a quantidade desejada e suas respectivas coordenadas.

\subsection{r\_point}
\textbf{Parâmetros de entrada :} Nenhum. \\
\textbf{Saída:} Retorna uma struct "a" com um par ordenado e uma flag setada em 0.\\

\textbf{Descrição:}
Gera dois pontos aleatóriamente distribuidos dentro do intervalo [0,1] e uma flag de inicialização com valor zero.

\subsection{resetFlag}
\textbf{Parâmetros de entrada :} Um vetor de structs do tipo point e N. \\
\textbf{Saída:} Nenhuma.\\

\textbf{Descrição:}
Executa um laço para resetar o valor das flags anteriormente utilizadas.

\section{Operation.c}
  Este módulo agrupa as funções de operação dos pontos e o algoritmo de conexidade.
  
\subsection{pointOutput}
\textbf{Parâmetros de entrada :} Um vetor de structs do tipo point e N. \\
\textbf{Saída:} Nenhuma.\\

\textbf{Descrição:}
A função passei pelo vetor de struct imprimindo as coordenadas em um arquivo de saida.

\subsection{Algoritmo de conexidade}\
Funções
\begin{enumerate}
\item[-] searchEdge 
\item[-] initVector
\item[-] distance.
\item[-] binRelat.
\item[-] arranjeList.
\end{enumerate}
\textbf{Entrada:} Um vetor de struct point e o tamanho N do vetor. \\
\textbf{Saída:} TRUE ou FALSE, dependendo da conexidade do grafo.\\

O algoritmo para teste de conexidade se baseia em um método nomeado por mim como "conexidade por raíz comum". Este método utiliza um vetor de inteiros de tamanho N para salvar um dada raiz e a correspondência de suas ramificações. \\ 
Considere o seguinte vetor:

\begin{table}[ht]
\centering
\begin{tabular}{c |c | c |c |c |c |c |c |c |c |c |c |c |c |c }
R & -1 & -1 & -1 & -1 & -1 & -1 & -1 & -1 & -1 & -1 & ... & -1 & -1 & -1 \\ \hline \hline
i & 0 & 1 & 2 & 3 & 4 & 5 & 6 & 7 & 8 & 9 & ... & i$_n-3$ & i$_n-2$ & i$_n-1$ 
\end{tabular}
\end{table}

Ele possui tamanho N, com indice indo de 0 até n-1. O valor de cada casela é setado em -1. O algoritmo começa sua análise no primeiro ponto da lista coordenada de gList[i]. O primeiro laço seta a coordenada i e passei através desta lista com o Índice k.

\begin{lstlisting}[frame=single]
  for(i = 0; i < N; i++){
        for (k = i+1; k < N; k++){
  .
  .
  .
\end{lstlisting}

Se i for conexo com k é atribuido inicialmente um d para i e este mesmo d é atribuido também para k. Por exemplo, se o d de  i for igual a 0 e for conexo com k = (2; 3; 5; 6) teremos:

\begin{table}[ht]
\centering
\begin{tabular}{c |c | c |c |c |c |c |c |c |c |c |c |c |c |c }
R & 0 & -1 & 0 & 0 & -1 & 0 & 0 & -1 & -1 & -1 & ... & -1 & -1 & -1 \\ \hline \hline
i & 0 & 1 & 2 & 3 & 4 & 5 & 6 & 7 & 8 & 9 & ... & i$_n-3$ & i$_n-2$ & i$_n-1$ 
\end{tabular}
\end{table}

O grafo pareceria com algo assim: \\
\begin{center}
\begin{tikzpicture}
  [scale=.8,auto=left,every node/.style={circle,fill=blue!20}]
  \node (n0) at (5,5) {0};
  \node (n2) at (4,4)  {2};
  \node (n3) at (2,6)  {3};
  \node (n5) at (8,5) {5};
  \node (n6) at (7,7)  {6};

  \foreach \from/\to in {n0/n2,n0/n3,n0/n5,n0/n6}
    \draw (\from) -- (\to);

\end{tikzpicture}
\end{center}

Considere agora que temos i como 1 e com ele são conexos k = (2; 8; 9), teremos a seguinte tabela:

\begin{table}[ht]
\centering
\begin{tabular}{c |c | c |c |c |c |c |c |c |c |c |c |c |c |c }
R & 0 & 1 & 0,1 & 0 & -1 & 0 & 0 & -1 & 1 & 1 & ... & -1 & -1 & -1 \\ \hline \hline
i & 0 & 1 & 2 & 3 & 4 & 5 & 6 & 7 & 8 & 9 & ... & i$_n-3$ & i$_n-2$ & i$_n-1$ 
\end{tabular}
\end{table}

e o grafo:

\begin{center}
\begin{tikzpicture}
  [scale=.8,auto=left,every node/.style={circle,fill=blue!20}]
  \node (n0) at (5,5)  {0};
  \node (n2) at (4,4)  {2};
  \node (n3) at (2,6)  {3};
  \node (n5) at (8,5)  {5};
  \node (n6) at (7,7)  {6};
  \node (n1) at (2,3)  {1};
  \node (n8) at (0,4)  {8};
  \node (n9) at (0,2)  {9};

  \foreach \from/\to in {n0/n2, n0/n3, n0/n5, n0/n6, n1/n2, n1/n8, n1/n9}
    \draw (\from) -- (\to);

\end{tikzpicture}
\end{center}

Inicialmente a raíz do grafo é o vértice 0. Como visto pelo teste, 0 não é conexo com 8 por vértice, mas sim conexo por caminhos. Como 1 ja era conexo inicialmente com 0 extendemos a conexidade com a raíz 0 para os novos vértices, que são 8 e 9. Atualizando a tabela teremos então:

\begin{table}[ht]
\centering
\begin{tabular}{c |c | c |c |c |c |c |c |c |c |c |c |c |c |c }
R & 0 & 1 & 0 & 0 & -1 & 0 & 0 & -1 & 0 & 0 & ... & -1 & -1 & -1 \\ \hline \hline
i & 0 & 1 & 2 & 3 & 4 & 5 & 6 & 7 & 8 & 9 & ... & i$_n-3$ & i$_n-2$ & i$_n-1$ 
\end{tabular}
\end{table}

Se i e k possuem uma mesma raíz eles são tidos como vizinhos, e para que isso aconteça estes devem ter o mesmo valor na tabela de conexidades nos índices i e k. Esta verificação é feita no seguinte trecho:

\begin{lstlisting}[frame=single]
  if(adjVector[i] == adjVector[k] && adjVector[i]!= -1)
    k++;
  .
  .
  .
\end{lstlisting}

,caso sejam vizinhos por conexidade direta ou por caminhos k é incrementado, passando para o próximo ponto da lista. Se não forem é testado se a distância entre i e k é menor ou igual a Edge:

 \begin{lstlisting}[frame=single]
    else if(distance(gList[i],gList[k]))
      binRelat(adjVector,&gList[i],&gList[k],i,k, N);
  .
  .
  .
\end{lstlisting}

se forem conexos a função binRelat teste se o ponto ja foi inicializado. Dependendo do caso, atribuições são feitas podendo a própria raíz mudar de posição, não ficando estática em 0. Considere o seguinte exemplo:

\begin{center}
\begin{tikzpicture}
  [scale=.8,auto=left,every node/.style={circle,fill=blue!20}]
  \node (n0) at (5,5)  {0};
  \node (n3) at (2,6)  {3};
  \node (n4) at (8,5)  {4};
  \node (n7) at (7,7)  {7};
  \node (n8) at (0,2)  {8};
  \node (n9) at (0,4)  {9};
  \node (n1) at (-4,2) {1};
  \node (n2) at (-3,1) {2};
  \node (n5) at (-2,2) {5};
  \node (n6) at (-1,1) {6};

  \foreach \from/\to in {n0/n3, n0/n7, n0/n4, n0/n9, n9/n8, n8/n6, n6/n5, n5/n2, n2/n1}
    \draw (\from) -- (\to);

\end{tikzpicture}
\end{center}

Inicialmente nossa raíz era o vértice 0, que teve k = (3; 4; 7; 9) e tabela:

\begin{table}[ht]
\centering
\begin{tabular}{c |c | c |c |c |c |c |c |c |c |c }
R & 0 & -1 & -1 & 0 & 0 & -1 & -1 & 0 & -1 & 0\\ \hline \hline
i & 0 & 1 & 2 & 3 & 4 & 5 & 6 & 7 & 8 & 9 
\end{tabular}
\end{table}

Com i igual a 1 houve apenas conexidade com k = (2), e sucessivamente até i igual e 6 e k = (8).

\begin{table}[ht]
\centering
\begin{tabular}{c |c | c |c |c |c |c |c |c |c |c }
R & 0 & 1 & 1 & 0 & 0 & 1 & 1 & 0 & 1 & 0\\ \hline \hline
i & 0 & 1 & 2 & 3 & 4 & 5 & 6 & 7 & 8 & 9 
\end{tabular}
\end{table}

Temos duas raízes e o grafo não está fechado. Na última comparação 8 é conexo com 9, ambas flags estão incializadas e eles não são vizinhos, analisando a função binRelat :
 \begin{lstlisting}[frame=single]
    if(!A->initFlag && !B->initFlag){
        A->initFlag = TRUE;
        B->initFlag = TRUE;
        adjVector[i] = i;
        adjVector[k] = i;
    }
    else{
        if(!B->initFlag){
            B->initFlag = TRUE;
            adjVector[k] = adjVector[i];
        }
        else{
            if(!A->initFlag){
               A->initFlag = TRUE;
               adjVector[i] = adjVector[k];
            }
            else{
                arrangeList(adjVector, adjVector[i]
                ,adjVector[k], N);
                }
        }
    }

\end{lstlisting}


entramos no terceiro else, na chamada da função arrangeList:
 \begin{lstlisting}[frame=single]
     int j;

    for (j = 0; j < N; j++)
        if (adjVector[j] == k)
            adjVector[j] = i;
            

\end{lstlisting}

Na função o valor de i, no caso 1, e atribui a todos os k's presente na lista, que são 0. O valor da nova raíz do grafo é 1 e a nova tabela é a seguinte: 

\begin{table}[ht]
\centering
\begin{tabular}{c |c | c |c |c |c |c |c |c |c |c }
R & 1 & 1 & 1 & 1 & 1 & 1 & 1 & 1 & 1 & 1\\ \hline \hline
i & 0 & 1 & 2 & 3 & 4 & 5 & 6 & 7 & 8 & 9 
\end{tabular}
\end{table}

Para um grafo ser conexo dados vértices i e k devem ter uma K raíz comum. Para verificar isto basta que o valor cada casela de índice i do vetor de raíz tenha um mesmo valor K. Caso duas caselas tenham valores diferentes o grafo não é conexo, exemplo:

\begin{table}[ht]
\centering
\begin{tabular}{c |c | c |c |c |c |c |c |c |c |c }
R & 0 & 1 & 1 & 0 & 0 & 1 & 1 & 0 & 1 & 1\\ \hline \hline
i & 0 & 1 & 2 & 3 & 4 & 5 & 6 & 7 & 8 & 9 
\end{tabular}
\end{table}

\begin{center}
\begin{tikzpicture}
  [scale=.8,auto=left,every node/.style={circle,fill=blue!20}]
  \node (n0) at (5,5)  {0};
  \node (n3) at (2,6)  {3};
  \node (n4) at (8,5)  {4};
  \node (n7) at (7,7)  {7};
  \node (n8) at (0,2)  {8};
  \node (n9) at (0,4)  {9};
  \node (n1) at (-4,2) {1};
  \node (n2) at (-3,1) {2};
  \node (n5) at (-2,2) {5};
  \node (n6) at (-1,1) {6};

  \foreach \from/\to in {n0/n3, n0/n7, n0/n4, n9/n8, n8/n6, n6/n5, n5/n2, n2/n1}
    \draw (\from) -- (\to);

\end{tikzpicture}
\end{center}

Se caso um dado vértice i ficar excluído de qualquer conexão o grafo também é considerado não conexo. Exemplo:

\begin{table}[ht]
\centering
\begin{tabular}{c |c | c |c |c |c |c |c |c |c |c }
R & 0 & 1 & 1 & -1 & 0 & 1 & 1 & 0 & 1 & 1\\ \hline \hline
i & 0 & 1 & 2 & 3 & 4 & 5 & 6 & 7 & 8 & 9 
\end{tabular}
\end{table}

\begin{center}
\begin{tikzpicture}
  [scale=.8,auto=left,every node/.style={circle,fill=blue!20}]
  \node (n0) at (5,5)  {0};
  \node (n3) at (2,6)  {3};
  \node (n4) at (8,5)  {4};
  \node (n7) at (7,7)  {7};
  \node (n8) at (0,2)  {8};
  \node (n9) at (0,4)  {9};
  \node (n1) at (-4,2) {1};
  \node (n2) at (-3,1) {2};
  \node (n5) at (-2,2) {5};
  \node (n6) at (-1,1) {6};

  \foreach \from/\to in { n0/n7, n0/n4, n9/n8, n8/n6, n6/n5, n5/n2, n2/n1}
    \draw (\from) -- (\to);

\end{tikzpicture}
\end{center}

Após análisar a conexidade de todos as coordenadas o trecho:

 \begin{lstlisting}[frame=single]
    connectIndx = adjVector[0];
    if (connectIndx != -1){
        for (i = 0; i < N; i++){
            if(connectIndx != adjVector[i]){
                conex = FALSE;
                break;
            }
        }
    }
\end{lstlisting}

verifica-se todos valores contidos na tabela e a igualdade com a raíz k. Se possuírem o mesmo K, o grafo é conexo e é retornado o valor lógico TRUE, caso um dos valores da lista seja diferente da raíz ou igual a -1, retorna-se FALSE.

\section{Graph.h}
Este header contem as duas estruturas de dados utilizadas ao longo do EP. A estrutura Grafo contem um ponteiro para struct point, que representa a lista de coordenadas de um Grafo G. Na estrutura point temos as coordenadas (x,y) de um vértice i definidas como variáveis flutuantes e uma flag de inicialização que ajuda durante o algoritmo de conexidade.

\end{document}
